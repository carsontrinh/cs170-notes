\chapter{Divide and Conquer}
\section{Master theorem}
Let \(T[n] = aT[n/b] + \Theta(n^d)\), \(a, b, d \in \mathbb{R}^+\).
\begin{description}
	\item \(d < \log_ba\)
\end{description}

\section{Matrix multiplication}
\subsection{Na\"ive algorithm}
If \(A \in k^{n\times n}\) and \(B \in k^{n\times n}\), then \(AB \in k^{n\times n}\), where \(\left(AB\right)_{ij} = A[i,:] \cdot B[:,j]\). There are \(n^2\) entries that take \(n\) time each, so computing \(AB\) na\"ively takes \(O(n^3)\) time.

\subsection{Divide-and-conquer algorithm}
Divide the matrices into blocks.
\begin{align}
	\begin{pmatrix}
		X_{ij}
	\end{pmatrix}
	\begin{pmatrix}
		Y_{ij}
	\end{pmatrix}
	&=
	\begin{pmatrix}
		\left(XY\right)_{ij}
	\end{pmatrix}\\
	\begin{pmatrix}
		A & B \\
		C & D
	\end{pmatrix}
	\begin{pmatrix}
		E & F \\
		G & H
	\end{pmatrix}
	&=
	\begin{pmatrix}
		AE + BG & AF + BH \\
		CE + DG &  CF + DH
	\end{pmatrix}
\end{align}

\begin{algorithm}
	\caption{Efficiently multiply two square matrices.}
	\begin{algorithmic}
		\Function{Mult}{$X, Y \in k^{n\times n}$}
			\State \(P_1 \leftarrow AE\)
			\State \(\ddots\) (6 steps omitted; see algebra above)
			\State \(P_8 \leftarrow DH\)
			\State \Return \(\begin{pmatrix}
				P_1 + P_2 & P_3 + P_4 \\
				P_5 + P_6 & P_7 + P_8
			\end{pmatrix}\)
		\EndFunction
	\end{algorithmic}
\end{algorithm}

Runtime analysis:
\begin{align}
	T[n] &= 8T\left[\frac{n}{2}\right] + O(n^2)
	\intertext{Apply Master Theorem.}
	&= O(n^3)
\end{align}

It turns out there's a way to end up with the results of those 8 submatrix multiplications using only 7 multiplications.
\begin{align}
T[n] &= \mathbf{7}T\left[\frac{n}{2}\right] + O(n^2) \\
&= O(n^{2.81})?
\end{align}

\section{Median}
The median of an \(n\)-set of numbers is the \(n/2\)th-largest number.

\subsection{Na\"ive algorithm}
Sort the list. \(O(n\log n)\).

\subsection{Divide-and-conquer algorithm}
Dividing into two lists and computing their medians doesn't always get you the median.
\begin{algorithm}
	\caption{Attempt at a divide-and-conquer algorithm for finding the median of a list of numbers \(n\) items long.}
	\begin{algorithmic}
		\Function{Median}{$S = \left\{a_1, \ldots, a_n\right\}$}
			\State Choose \(v \in S\)
			\State \(S_L \leftarrow \left\{a \in S \mid a < v \right\}\)
			\State \(S_V \leftarrow \left\{a \in S \mid a = v \right\}\)
			\State \(S_R \leftarrow \left\{a \in S \mid a > v \right\}\)
			\State \Return ?? to be continued
		\EndFunction
	\end{algorithmic}
\end{algorithm}

It is helpful to generalize median to selection.
\begin{algorithm}
	\caption{Select \(k\)th smallest number in a set.}
	\begin{algorithmic}
		\Function{Select}{$S = \left\{a_1, \ldots, a_n\right\}$, k}
		\State Choose \(v \in S\)
		\State \(S_L \leftarrow \left\{a \in S \mid a < v \right\}\)
		\State \(S_V \leftarrow \left\{a \in S \mid a = v \right\}\)
		\State \(S_R \leftarrow \left\{a \in S \mid a > v \right\}\)
		\If{\(\left|S_L\right| > k\)}
			\State \Return \(\text{Select}(S_L, k)\)
		\EndIf
		\If{\(k > \left|S_L\right|\) and \(k \leq \left|S_L\right| + \left|S_V\right|\)}
			\State \Return \(v\)
		\EndIf
		\If{\(k > \left|S_L\right| + \left|S_V\right|\)}
			\State \Return \(\text{Select}(S_R, k - \left|S_L\right| - \left|S_V\right|)\)
		\EndIf
		\EndFunction
	\end{algorithmic}
\end{algorithm}

\subsection{Runtime analysis}
In the worse case, this algorithm chooses an extreme every single call. So \(\Theta(n)\) times you have to separate \(\Theta(n)\) items and the runtime is \(\Theta(n^2)\).

How shall we judge the average case? (It is not interesting to consider only the best case, which is \(\Theta(1)\).) Instead, call pivots \emph{good} which are within \(n/4\) places of the median. The following inequalities follow:
\begin{align}
\left|S_L\right| &\leq \frac{\left|S\right|}{4} \\
\left|S_R\right| &\leq \frac{\left|S\right|}{4}
\end{align}
A ``good'' pivot reduces the problem size by \(\frac{3}{4}\) per stack frame.

Now to consider probability. Any given step,
\begin{align}
	\Pr\left(\text{\(v\) is a good pivot}\right) &= \frac{1}{2}
	\intertext{Rewrite the runtime expression.}
	T[n] &= \text{time after first good pivot} \notag\\&\phantom{=} + \text{time until first good pivot}\\
	&\leq T\left[\frac{3}{4}n\right] + n\text{E}\left[\text{\# of pivots}\right] \\
	\intertext{``trust me, this is 2.''}
	&\leq T\left[\frac{3}{4}n\right] + 2n\\
	&\leq O(n)
\end{align}